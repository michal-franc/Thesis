\chapter{Wybrane aspekty projektowania}
!!
\section{Wymagania u�ytkowe}

- Grupy (tematyczne , szkoleniowe)

System powinien posiadac mechanizm pozwalajacy na tworzenie grup learningowych by 
skupiac uzytkownikow podzial taki pozwala osobie prowadzacej szkolenie latwiej kontrolowac
przeplyw informacji. Tworzenie grup tematycznych dodatkowo pozwala skupic osoby o podobnych zainteresowaniach.

- Chat Grupowy ala twitter z zostawianymi wiadomosciami badz shoutbox

- Forum grupowe 
  Proste Forum

- Kalendarz grupowy
Mozliwosc dodawania wydarzen (testy , spotkania online , terminy)

-Profile

Kazda osoba korzystajaca z serwisu musi posiadac Profil. Kazdy uzytownik moze edytowac swoje dane
osobowe. 

Uzytkownik posiada pelna informacje do jakich grup jest zapisany oraz w jakich kursach bierze udzial
Moze przegladac nadchodzace wydarzenia i nowosci zwiazane z jego grupa szkoleniowa.

Szkoleniowiec moze sledzic swoje grupy i postepy uczestnikow na jego kursach oraz testach.
Posiada takze wglad w roznego rodzaju statystyki aktywnosci , zdawalnosci itd

-Wewnetrzny system prywatncyh wiadomosci

System pozwalajacy komunikowac sie asynchronicznie uzytkownikom.

- Oceny - system wyswatwiania ocen i rpzegladania

Kursy moga miec mozliwosc wystawiania ocen na podstawie mechanizmu tworzenia testow



- chat

grupy posiadaja swoej chaty w postaci mikro blogow

- moderacja zabezpieczeniami - szkoleniowiec moze udostepniac roznoraki dostep dla uzytkownikow kursow
mozna np wylaczyc chat dla pewnych osob

- komunikacja glosowa ?
- video ?

- Narzedzia wspomagajace proces tworzenia materialow elearningowych
	- Mechanizm tworzenia testow

	proste pytanie odpoowiedz do bardziej interkatywnych zabaw np dla programowania parser / kompilator dla 
	paru popularnych jezykow

	- tworzenie screencastow

	- wrzucanie filmikow video 

	- tworzenie prostych prezentacji
	- tworzenie prostych diagramow

- mozliwosc wczytywania zestandaryzowanych e learningowych paczek
- modu� raport�w i statystyk
- przy bardziej z�o�onych kursach �cie�ka kszta�cenia
- system badajacya predyspozycje kursanta i polecajacy nastepny kurs
-dodatkowo w platformie laczacej rownie� system ERP i kompetencje pracownicze
 mo�na zrobi� modu� rekomendacji pracowniczych oceny pracownik�w , narzedzia wspomagajace planowanie szkolen dla zespolow pro

\section{Architektura system�w zdalnego nauczania}

Baza Danych + Logika + UI
\section{Sposoby komunikacji z u�ytkownikiem ko�cowym}
!!
\section{Bezpiecze�stwo systemu}
!!
\section{SOA - Architektura zorientowana na us�ugi}
!!
\section{Projektowanie us�ug sieciowych w �rodowisku .Net}
!!
\section{Projekt systemu}
!!
\subsection{Model logiczny i koncepcja systemu}
!!
\subsection{Schemat przep�ywu danych}
!!
\subsection{Funkcje systemu}
!!
\subsubsection{Diagram przypadk�w u�ycia}
!!
\subsubsection{Sceniariusze wybranych przypadk�w}
!!
\subsection{Diagram klas systemu}
!!
\subsection{Projekt bazy danych}
!!
\subsubsection{Model logiczny}
!!
\subsubsection{Model fizyczny}
!!
\subsubsection{Framework wspomagaj�cy tworzenie us�ug sieciowych korzystaj�cych z bazy za pomoc� ORM-a}
!!
\subsection{Algorytmy i metody przetwarzania danych z wykorzystaniem us�ug sieciowych}
!!
\subsection{Protok� komunikacyjny w �rodowisku rozproszonych us�ug sieciowych}
!!
\subsection{Projekt systemu zabezpiecze�}
!!
\subsubsection{Mechanizm oparty na us�ugach sieciowych z integrowany z protoko�em OpenID}
